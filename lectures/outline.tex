% Options for packages loaded elsewhere
\PassOptionsToPackage{unicode}{hyperref}
\PassOptionsToPackage{hyphens}{url}
\PassOptionsToPackage{dvipsnames,svgnames,x11names}{xcolor}
%
\documentclass[
]{article}
\usepackage{amsmath,amssymb}
\usepackage{iftex}
\ifPDFTeX
  \usepackage[T1]{fontenc}
  \usepackage[utf8]{inputenc}
  \usepackage{textcomp} % provide euro and other symbols
\else % if luatex or xetex
  \usepackage{unicode-math} % this also loads fontspec
  \defaultfontfeatures{Scale=MatchLowercase}
  \defaultfontfeatures[\rmfamily]{Ligatures=TeX,Scale=1}
\fi
\usepackage{lmodern}
\ifPDFTeX\else
  % xetex/luatex font selection
\fi
% Use upquote if available, for straight quotes in verbatim environments
\IfFileExists{upquote.sty}{\usepackage{upquote}}{}
\IfFileExists{microtype.sty}{% use microtype if available
  \usepackage[]{microtype}
  \UseMicrotypeSet[protrusion]{basicmath} % disable protrusion for tt fonts
}{}
\makeatletter
\@ifundefined{KOMAClassName}{% if non-KOMA class
  \IfFileExists{parskip.sty}{%
    \usepackage{parskip}
  }{% else
    \setlength{\parindent}{0pt}
    \setlength{\parskip}{6pt plus 2pt minus 1pt}}
}{% if KOMA class
  \KOMAoptions{parskip=half}}
\makeatother
\usepackage{xcolor}
\usepackage[margin=1in]{geometry}
\usepackage{graphicx}
\makeatletter
\def\maxwidth{\ifdim\Gin@nat@width>\linewidth\linewidth\else\Gin@nat@width\fi}
\def\maxheight{\ifdim\Gin@nat@height>\textheight\textheight\else\Gin@nat@height\fi}
\makeatother
% Scale images if necessary, so that they will not overflow the page
% margins by default, and it is still possible to overwrite the defaults
% using explicit options in \includegraphics[width, height, ...]{}
\setkeys{Gin}{width=\maxwidth,height=\maxheight,keepaspectratio}
% Set default figure placement to htbp
\makeatletter
\def\fps@figure{htbp}
\makeatother
\setlength{\emergencystretch}{3em} % prevent overfull lines
\providecommand{\tightlist}{%
  \setlength{\itemsep}{0pt}\setlength{\parskip}{0pt}}
\setcounter{secnumdepth}{-\maxdimen} % remove section numbering
%% MISC
\usepackage{graphicx}
\usepackage{url}
\usepackage[colorlinks=true, urlcolor = orange, linkcolor=blue, citecolor=purple]{hyperref}
\usepackage[utf8]{inputenc}
%\usepackage[margin=1.5in]{geometry}
\usepackage{amssymb}
\usepackage{amsmath}
\usepackage{amsfonts}
%\usepackage{fontspec}

%% FONTS
\usepackage{charter} 
\usepackage[expert]{mathdesign}

%\frenchspacing
\usepackage{setspace}
\setstretch{1.1}

\usepackage{booktabs}
\usepackage{siunitx}
\ifLuaTeX
  \usepackage{selnolig}  % disable illegal ligatures
\fi
\usepackage{bookmark}
\IfFileExists{xurl.sty}{\usepackage{xurl}}{} % add URL line breaks if available
\urlstyle{same}
\hypersetup{
  colorlinks=true,
  linkcolor={purple},
  filecolor={Maroon},
  citecolor={Blue},
  urlcolor={purple},
  pdfcreator={LaTeX via pandoc}}

\title{Prospective Outline\\
Empirical Economics\\
2025-2026}
\author{Bas Machielsen\\
\href{mailto:a.h.machielsen@uu.nl}{\nolinkurl{a.h.machielsen@uu.nl}}}
\date{September 2025}

\begin{document}
\maketitle

\subsection{Prospective Outline Empirical
Economics}\label{prospective-outline-empirical-economics}

\subsubsection{Lecture 1: Statistics \&
Probability}\label{lecture-1-statistics-probability}

\begin{enumerate}
\def\labelenumi{\arabic{enumi}.}
\item
  Basic Probability Theory: Experiments, Outcomes, Sample Spaces,
  Events.
\item
  Conditional Probability: \(P(A|B) = P(A \cap B) / P(B)\).
\item
  Random Variables: Definition: A variable whose value is a numerical
  outcome of a random phenomenon. Discrete Random Variables
  vs.~Continuous Random Variables.
\item
  Probability Distributions for Discrete Random Variables: Probability
  Mass Function (PMF). Expected Value (\(E[X]\)) or Mean of a discrete
  random variable. Variance (\(Var(X)\)) and Standard Deviation of a
  discrete random variable. Example with Bernoulli Distribution.
\item
  Probability Distributions for Continuous Random Variables: 6.1
  Probability Density Function (PDF) -- area under the curve represents
  probability. 6.2 Cumulative Distribution Function (CDF) for both
  discrete and continuous variables.
\end{enumerate}

6The Normal Distribution: Properties: Bell-shaped, symmetric, defined by
mean and variance The Standard Normal Distribution (Z-distribution)
Using Z-tables or software to find probabilities.

\begin{enumerate}
\def\labelenumi{\arabic{enumi}.}
\setcounter{enumi}{6}
\item
  Covariance and Correlation: Covariance: Measuring the direction of
  linear relationship. Correlation Coefficient (ρ or r): Measuring
  strength and direction of linear relationship (-1 to +1). Distinction:
  Correlation does not imply causation.
\item
  Introduction to Sampling: Population vs.~Sample. Parameters
  (population characteristics) vs.~Statistics (sample characteristics).
  Simple Random Sampling.
\item
  The Concept of a Sampling Distribution: The distribution of a sample
  statistic (e.g., sample mean) if we were to draw many samples.
\item
  The Central Limit Theorem (CLT): Statement and profound implications
  for the sampling distribution of the sample mean. Importance for
  inference even when the population is not normal (for large enough
  sample sizes).
\item
  Introduction to Estimation: Point Estimators (e.g., sample mean
  ̄\(\bar{x}\) as an estimator for population mean \(\mu\). Desirable
  properties of estimators (unbiasedness, efficiency, consistency -
  conceptual).
\item
  Introduction to Hypothesis Testing (Conceptual): Null Hypothesis
  \(H_0\) and Alternative Hypothesis \(H_A\) The idea of test statistics
  and p-values (without formal calculations yet).
\end{enumerate}

\subsubsection{Lecture 2: The Linear
Model}\label{lecture-2-the-linear-model}

\begin{enumerate}
\def\labelenumi{\arabic{enumi}.}
\item
  What is Econometrics? Why study it? The nature of economic data:
  Cross-sectional, Time Series, Pooled Cross-sections, Panel Data. The
  concept of a model: Population Regression Function (PRF).
\item
  The Sample Regression Function (SRF). 2.1 Derivation of Ordinary Least
  Squares (OLS) estimators for SLR (minimizing Sum of Squared Residuals
  - SSR). 2.2 Algebraic properties of OLS statistics (fitted values,
  residuals). 2.3 Interpreting the intercept \(\beta_0\) and \(\beta_1\)
  coefficients in OLS 2.4 Units of measurement and functional form
  (level-level). 2.5 Goodness-of-Fit: R-squared and Standard Error of
  the Regression (SER). 2.6 Understanding Statistical Output
\item
  Classical Assumptions 3.1 No perfect collinearity. 3.2 Zero
  conditional mean of errors \(E(u|x) = 0\) for unbiasedness 3.3
  Unbiasedness of OLS estimators 3.4 Variance of OLS estimators
\item
  Introduction to Multiple Linear Regression 4.1 OLS estimation of the
  MLR model
\end{enumerate}

\subsubsection{Lecture 3: Time Series and
Prediction}\label{lecture-3-time-series-and-prediction}

\subsubsection{Lecture 4: Panel Data}\label{lecture-4-panel-data}

What is Panel Data? Structure (\(N\) individuals/entities, \(T\) time
periods). Advantages of Panel Data: Controlling for unobserved
heterogeneity, increased degrees of freedom, dynamics. Notation for
panel data models. The Pooled OLS Model on panel data: Assumptions and
strong limitations (ignores heterogeneity). Unobserved Heterogeneity:
Individual-specific effects (\(\alpha_i\)) and time effects
(\(\lambda_t\)). The Fixed Effects (FE) Model: Treating \(\alpha_i\) as
parameters to be estimated. The ``Within'' Estimator: Transformation via
de-meaning data. The Least Squares Dummy Variable (LSDV) Estimator:
Equivalence to Within estimator. Interpretation of coefficients in FE
models (effect of X on Y within individuals over time). Pros and Cons of
FE: Controls for time-invariant unobserved heterogeneity correlated with
X, but cannot estimate effects of time-invariant variables.

The Random Effects (RE) Model: Treating \(\alpha_i\) as part of the
composite error term. Key Assumption for RE: \(E(\alpha_i|X_{it}) = 0\)
(unobserved heterogeneity is uncorrelated with regressors). Estimation
of RE models (Generalized Least Squares - GLS, or Feasible GLS - FGLS).
Interpretation of coefficients in RE models. Pros of RE: Can estimate
effects of time-invariant variables, more efficient than FE if
assumptions hold. Cons of RE: Biased if key
assumption\(E(\alpha_i|X_{it}) = 0\) is violated. Comparing FE and RE:
The Hausman Test - intuition and application. Null hypothesis of Hausman
test (RE is consistent and efficient) and decision rule. First
Differences (FD) Estimator: Another way to eliminate fixed effects;
comparison with FE. Practical considerations: Balanced vs.~unbalanced
panels, choosing between Pooled OLS, FE, and RE.

\subsubsection{Lecture 5: Binary
Outcome}\label{lecture-5-binary-outcome}

Introduction to Binary (Dummy) Dependent Variables (e.g., employment,
purchase decision). The Linear Probability Model (LPM): Applying OLS to
a binary dependent variable. Interpretation of coefficients in LPM as
changes in probability. Advantages of LPM: Simplicity of estimation and
interpretation. Problem 1 with LPM: Predicted probabilities can be
outside \([0,1]\). Problem 2 with LPM: Non-normality of the error term.
Problem 3 with LPM: Inherent heteroskedasticity of the error term. The
need for non-linear models for binary outcomes. Introduction to the
Latent Variable framework as motivation for Probit/Logit. The Probit
Model: Assumption of a normally distributed latent error term; use of
the Normal CDF. The Logit Model: Assumption of a logistically
distributed latent error term; use of the Logistic CDF. Estimation:
Maximum Likelihood Estimation (MLE) -- intuition and basic principles.
Interpretation of Coefficients: Raw coefficients are not directly
marginal effects. Calculating and interpreting Marginal Effects (MEs):
Average Marginal Effects (AMEs) vs.~Marginal Effects at Means (MEMs).
Odds Ratios: Specific interpretation for Logit models. Goodness-of-Fit
for Probit/Logit: Pseudo R-squared (e.g., McFadden's), percent correctly
predicted, likelihood ratio tests. Choosing between Probit and Logit
(often similar results, logistic distribution has fatter tails).

\subsubsection{Lecture 6:
Difference-in-differences}\label{lecture-6-difference-in-differences}

Recap: Correlation vs.~Causation -- the fundamental challenge. The
Potential Outcomes Framework (Rubin Causal Model): \(Y_i (1), Y_i (0)\)
The concept of the Average Treatment Effect (ATE). The Fundamental
Problem of Causal Inference: We only observe one potential outcome per
unit. Selection Bias: Why simple comparisons of treated and untreated
groups can be misleading. Introduction to Differences-in-Differences
(DiD): Intuition and basic idea. The 2x2 DiD Setup: Two groups
(Treatment, Control) and two time periods (Before, After). Calculating
the simple DiD estimator:
\((\hat{Y}_{T,Post} - \hat{Y}_{T, Pre}) - (\hat{Y}_{C, Post} - \hat{Y}_{C,Pre})\).
The ``Parallel Trends'' Assumption: The crucial identifying assumption
for DiD. Graphical illustration of the parallel trends assumption and
the DiD estimate. DiD using a Regression Framework:
\(Y_{it} = \beta_0 + \beta_1 Treat_i + \beta_2 Post_t + β_3 (Treat_i \times Post_t) + \epsilon_{it}\)
Interpretation of coefficients in the DiD regression Advantages of the
regression framework (SEs, inclusion of covariates). Adding Covariates
to the DiD model: Controlling for observable differences that might
affect trends. Testing the Parallel Trends Assumption: Pre-treatment
trend checks (visual, statistical tests on pre-treatment periods if data
allows). DiD with multiple time periods and staggered adoption
(conceptual overview). Event Study plots as a common way to visualize
DiD with multiple periods. Potential Pitfalls and Limitations of DiD
(e.g., Ashenfelter's dip, policy anticipation, spillover effects).
Robustness Checks for DiD studies (e.g., alternative control groups,
placebo tests).

\subsubsection{Lecture 7: Propensity Score Matching,
IV}\label{lecture-7-propensity-score-matching-iv}

\subsubsection{Lecture 8: Regression
Discontinuity}\label{lecture-8-regression-discontinuity}

\end{document}
